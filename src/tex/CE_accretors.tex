\documentclass[twocolumn,twocolappendix,trackchanges]{aastex63}
\usepackage{amsmath}
\usepackage{showyourwork}
\newcommand{\code}[1]{\texttt{#1}}
\newcommand{\mesa}{\code{MESA}}
\newcommand{\MESA}{\code{MESA}}
\renewcommand{\labelitemii}{$\bullet$}
\newcommand{\kms}{{\mathrm{km\ s^{-1}}}}
\newcommand{\kev}{\mathrm{keV}}
\newcommand{\gk}{\ensuremath{\,\rm{GK}}}
\usepackage{CJK}
\DeclareRobustCommand{\Eqref}[1]{Eq.~\ref{#1}}
\DeclareRobustCommand{\Figref}[1]{Fig.~\ref{#1}}
\DeclareRobustCommand{\Tabref}[1]{Tab.~\ref{#1}}
\DeclareRobustCommand{\Secref}[1]{Sec.~\ref{#1}}

\newcommand{\todo}[1]{{\large $\blacksquare$~\textbf{\color{red}[#1]}}~$\blacksquare$}

\newcommand{\Msun}{\ensuremath{\,M_\odot}}
\newcommand{\Lsun}{\ensuremath{\,L_\odot}}
\begin{document}

\graphicspath{{./figures/}}

\title{Rejuvenated stars are more easily unbound:\\ The first stable
  mass transfer impacts the donor of common envelopes in gravitational wave progenitors}

\author[0000-0002-6718-9472]{M.~Renzo \todo{other and order TBD}}
\affiliation{Center for Computational Astrophysics, Flatiron Institute, New York, NY 10010, USA}

% % Potential co-authors:
\author[0000-0001-5228-6598]{K.~Breivik}
\affiliation{Center for Computational Astrophysics, Flatiron Institute, New York, NY 10010, USA}

\author[0000-0003-3441-7624]{R.~Farmer}
\affiliation{Max-Planck-Institut für Astrophysik, Karl-Schwarzschild-Straße 1, 85741 Garching, Germany}
% E.~Zapartas
% Y.~Gotberg
\author[0000-0002-6592-2036]{M. Lau}
\affiliation{School of Physics and Astronomy, Monash University,
  Clayton, Victoria 3800, Australia}
\affiliation{OzGrav: The ARC Centre of Excellence for Gravitational Wave Discovery, Australia}
\affiliation{Center for Computational Astrophysics, Flatiron  Institute, New York, NY 10010, USA}
\author{S.~Justham}
\affiliation{Anton Pannekoek Institute of Astronomy and GRAPPA,
  University of Amsterdam, Science Park 904, 1098 XH Amsterdam, The
  Netherlands}
\affiliation{School of Astronomy and Space Science, University of the Chinese Academy of Sciences, Beijing 100012, PR China}
\author[0000-0002-4670-7509]{B.~D.~Metzger}
\affiliation{Columbia Astrophysics Laboratory, Columbia University, New York, New York 10027, USA}
\affiliation{Center for Computational Astrophysics, Flatiron Institute, New York, NY 10010, USA}
% M. Cantiello

% \author[0000-0002-6960-6911]{Y.~G\"otberg}\thanks{Hubble Fellow}
% \affiliation{The Observatories of the Carnegie Institution for
%  Science, 813 Santa Barbara Street, Pasadena, CA 91101, USA}


\begin{abstract}
  Common envelope (CE) evolution is an outstanding open problem in
  stellar evolution and it is critical to the formation of compact
  binaries, including gravitational-wave sources. In the ``classical''
  isolated binary evolution scenario for double compact objects, the
  CE phase is usually the second mass transfer. Thus, the donor star
  of the CE is the product of a previous binary interaction, often a
  stable Roche lobe overflow (RLOF). Because of the accretion of mass
  during the first RLOF, the convective core of the accretor star
  grows and is ``rejuvenated''. This modifies the core-envelope
  boundary region and decreases significantly the envelope binding
  energy. Comparing accretor models from self-consistent binary models
  to stars evolved as single, we demonstrate that the ``rejuvenation''
  can lower the amount of energy necessary for a CE ejection for both
  black hole and neutron star progenitors by $\sim 4-58\%$ depending
  on the evolutionary stage and final orbital separation. Despite
  their high mass, our accretors also experience extended ``blue
  loops'' , which may have observational consequences for
  low-metallicity stellar populations and asteroseismology. Extending
  our small grid will be necessary to produce a rapid population
  synthesis algorithm to account for this effect.
  % tex-count-words = 190
\end{abstract}

\keywords{stars: massive -- stars: binaries: mass transfer -- stars:
  binaries: common envelope -- stars: binaries: accretors}

\section{Introduction}
\label{sec:intro}

Common envelope (CE) is an important evolutionary channel for
massive isolated binaries to become gravitational-wave (GW) sources, despite
recent debates on its relevance for the progenitors of the most
massive binary black holes \citep[e.g.,][]{pavlovskii:2017,
  klencki:2020, klencki:2021, vanson:2021, marchant:2021}.
CE remains a crucial step in the
formation, among many other compact binaries, of cataclysmic variable
\citep{paczynski:1976}, double white dwarfs
\citep[e.g.,][]{zorotovic:2010, korol:2017, kremer:2017, renzo:21gwce,
  thiele:21}, binary neutron stars
\citep[NS, e.g.,][]{vigna-gomez:2018, vigna-gomez:2020}, merging black hole-neutron stars
\citep[e.g.,][]{kruckow:18, broekgaarden:21} and possibly low-mass binary black
holes \citep[BH, e.g.,][]{dominik:2012, vanson:2021}.

In the ``classical scenario'' for binary BHs and/or NSs
\citep[e.g.,][]{tutukov:93,belczynski:2016, tauris:2017}, the
progenitor binary % gravitational-wave
% progenitor system
experiences a first dynamically stable mass transfer
phase through Roche lobe overflow (RLOF) between two non-compact
stars. Subsequently, the initially more massive RLOF-donor collapses to a
compact object without disrupting the binary
\citep[e.g.,][]{blaauw:1961,renzo:2019walk}. Only afterwards, as the
initially less massive RLOF-accretor expands, a second mass-transfer phase
occurs and is typically dynamically unstable, that is a CE
\citep[e.g.,][]{dominik:2012, belczynski:2016, kruckow:18}. This
second mass transfer is responsible for the orbital shrinking
\citep{paczynski:1976} allowing the system to merge within the age of
the Universe. Therefore, in this scenario, the donor star of the CE is
the former accretor of the first RLOF \citep[e.g.,][]{klencki:2020,
  law-smith:2020, renzo:2021zoph}.

The first stable RLOF typically occurs during the main sequence of the
initially less massive star and accretion modifies its structure
\citep[e.g.,][]{neo:1977, packet:1981, blaauw:1993, cantiello:2007,
  renzo:2021zoph}. On top of the enrichment of the envelope with
CNO-processed material from the donor star core \citep{blaauw:1993,
  renzo:2021zoph, el-badry:2022a}, and the substantial spin-up,
accretors are expected to adjust their core-size to the new mass in a
``rejuvenation'' process \citep[e.g.,][]{neo:1977, hellings:1983,
  hellings:1984}. The readjustment is driven by mixing at the boundary
between the convective core and the envelope, which refuels the
convective region of hydrogen (H). This mixing also affects the
thermal structure of the layer above the helium (He) core partially
depleted in H \citep[][]{renzo:2021zoph}, which we refer to as core-envelope
boundary (CEB) layer. It is in the CEB that the density rises and most
of the envelope binding energy is accumulated for the remaining
stellar lifetime. Consequently, the success or failure of
the CE ejection, which is likely decided in the CEB layer
\citep[e.g.,][]{ivanova:2013}, may be different depending on whether
the CE-donor accreted mass previously or not.

\todo{fix below}
Here, we use structure and evolution binary models to study the impact
of the first RLOF phase on the outcome of possible subsequent CE
events. \Secref{sec:methods} describe our \mesa\ calculations, and
\Secref{sec:intro_res} presents a proof-of-principle of the effect of
core boundary mixing and rotation on the envelope binding energy. In
\Secref{sec:bin_models} we compare our accretor models to single stars
with the same total post-RLOF mass. We discuss our findings and
conclude in \Secref{sec:conclusions}.

\section{Pre-common envelope evolution}
\label{sec:methods}

We use \mesa\ \citep[version 15140,][]{paxton:2011, paxton:2013,
  paxton:2015, paxton:2018, paxton:2011} to compute the evolution of
binaries which experience mass transfer after the end of the donor's
main sequence, that is case B Roche lobe overflow
\citep[RLOF,][]{kippenhahn:1967}. Our setup is similar to \cite{renzo:2021zoph},
except for the metallicity: here we adopt $Z=0.0019\simeq Z_\odot/10$,
relevant for the progenitor population of GW events
\citep[e.g.,][]{vanson:2021}. Moreover, we apply throughout the star a
small amount of mixing with diffusivity \texttt{min\_D\_mix}=100. This
improves the numerical stability by smoothing properties across
adjacent cells, without introducing significant quantitative
variations, and is a typical numerical technique used in
asteroseismology calculations (J.~Fuller, private~comm.).

We adopt initial period $P=100$\,days and choose initial masses
$(M_{1}, M_{2}) = (18, 15), (20, 17), (38, 30)\,M_\odot$. We focus on
the initially less massive stars, which after accretion become
$M_2=15\rightarrow 18, 17\rightarrow 20, 30\rightarrow 36\,M_\odot$,
roughly\footnote{Whether a core-collapse results in NS or BH
  formation, with or without an associated explosion, cannot be
  decided solely based on the (total or core) mass of a star
  \citep[e.g.,][]{oconnor:11, farmer:16, patton:20, zapartas:21b,
    patton:22}. } representative of NS, uncertain core-collapse
outcome, and BH progenitors, respectively. Our \mesa\ models assume
that the accretion efficiency is limited by rotationally enhanced wind
mass loss \citep[e.g.,][]{sravan:2019, wang:2020, renzo:2021zoph,
  sen:2022}. However, this may lead to less conservative mass transfer
than suggested by observations \citep[e.g.,][]{wang:2021a}.

We reduce the computing time and complexity by avoiding to compute the
late evolutionary phases of the RLOF-donors. After the binary detaches
from the Roche lobe (see discussion in \citealt{renzo:2021zoph}), our
simulations artificially detach\footnote{We make the routine to detach
  a \mesa\ binary on-the-fly publicly available
  \url{https://github.com/MESAHub/mesa-contrib/}} the stars and
continue the evolution of the accretor as a single star until it
reaches carbon depletion (defined by central carbon mass fraction
$X_\mathrm{c}(^{12}\mathrm{C})<2\times10^{-4} $). This implies that we
neglect further -- possible, but not expected -- mass transfer
episodes (case BB RLOF, \citealt{laplace:2020}) and tidal interactions
between the stars. We also neglect the impact of the SN ejecta with
the accretor \cite[e.g.,][]{hirai:2018, ogata:2021} and the orbital
consequences of the RLOF-donor core-collapse
\citep[e.g.,][]{brandt:1995, kalogera:1996, tauris:1998,
  renzo:2019walk}.

To illustrate the physical reason why the first RLOF may influence the
envelope structure of the accretor much later on, we also compute
comparison stars. For each mass, we compute non-rotating single stars
with otherwise identical setup, and ``engineered'' stars which we
modify at terminal age main sequence (TAMS, central hydrogen mass
fraction $X_\mathrm{c}(^1\mathrm{H})<10^{-4}$) to mimic the impact of
rejuvenation of the accretors CEB (see also Appendix~\ref{sec:toy_models}).

We compare the internal structure of accretors to single and
engineered stars at various radii
($R=100, 200, 300, 500, 1000\,R_\odot$). At the onset of a CE event,
$R\equiv R_\mathrm{RL, donor}$ is the size of the Roche lobe of the
donor star determined by the binary separation and mass ratio
\citep[e.g.,][]{paczynski:1971, eggleton:83}.% as it becomes a red
% supergiant.

All of our input files are available at \url{10.5281/zenodo.6600641}
and our output is compatible with the population synthesis code
POSYDON \citep{fragos:2022} and publicly available at \todo{fix}.



\section{Accretors from self-consistent binary models}
\label{sec:bin_models}
\todo{smooth}
We now consider accretors computed self-consistently in a binary until
detachment, and subsequently as single stars until carbon core
depletion. \Figref{fig:HRD} shows the evolution of our binaries on the
Hertzsprung-Russell (HR) diagram. The thin dashed lines % in
% \Figref{fig:HRD}
show the evolution of the donor stars from ZAMS,
through RLOF, until our definition of detachment \citep[see, e.g.,][]{yoon:2017,
  gotberg:2017, gotberg:2018, laplace:2020, laplace:2021}. The solid
lines correspond to the full evolution of the accretors, from zero age
main sequence (ZAMS), through RLOF (see \citealt{sravan:2019,
  renzo:2021zoph, wang:2020}), until carbon depletion. % We analyze the
% RLOF-accretor/CE-donor models at given photospheric radii
% $R=100, 200, 300, 500, 1000\,R_\odot$, corresponding to the Roche
% radius of the CE-donor, marked by gray dot-dashed lines in \Figref{fig:HRD}.

\begin{figure}[htbp]
  \includegraphics[width=0.5\textwidth]{HRD.pdf}
  \caption{HR diagram of the binary systems. The thin dashed lines
    show the evolution of the donors until RLOF detachment, the solid
    lines show the accretors from ZAMS, through RLOF accretion, until
    core carbon depletion. Thin dot-dashed lines mark constant radii
    of $R=100, 200, 300, 500, 1000\,R_\odot$, all models have
    $Z=0.0019$, period $100$\,days, and initial masses of 38 and
    30\,$M_\odot$ (green), 20 and 17\,$M_\odot$ (orange), 18 and
    15\,$M_\odot$ (blue). \todo{annotate key epochs?}}
  \label{fig:HRD}
  \script{HRD.py}
\end{figure}

% Regardless of their mass
% ($\gtrsim 15\,M_\odot$),
All three accretor models experience a blueward evolution after
beginning to ascend the Hayashi track. In the two lowest mass models,
this results in a blue-loop, which last $\sim{}10^5$ years. These two
models spend a significant fraction of their He core burning as hot
yellow/blue supergiants, and reach
$\log_{10}(T_\mathrm{eff}/\mathrm{[K]})\gtrsim 4.2$. Our most massive
  accretor ($M\simeq36\,M_\odot$) evolves towards hotter temperatures
  during core He burning, but never fully recovers closing the blue
  loop. Its excursion to hottest temperatures occurs after He core
  depletion and lasts $\sim{}10^{4}$ years for the most massive one.

Blue loops are not expected for single stars with
$M\gtrsim 12\,M_\odot$ \citep[e.g.,][]{walmswell:2015}, and their
occurrence is known to be sensitive to the He profile above the
H-burning shell, and specifically the mean molecular weight profile
\citep{walmswell:2015, farrell:22}. Thus it is not surprising that
RLOF-accretion, which modifies the stellar profile above the H-shell,
may lead to blue loops, and formation of yellow-supergiants
\citep[e.g.,][]{dorn-wallenstein:20}. We note that comparison single
stars also experience a late blue-ward evolution, but not a ``loop''
back to red. This behavior is likely related to the relatively high
wind mass-loss rate assumed (see \citealt{renzo:2017}), and the models
with initial mass $\gtrsim 30\,M_\odot$ are qualitatively similar to
the most massive accretor in \Figref{fig:HRD} even without accreting
matter from a companion: the occurrence of blue loops is notoriously
sensitive to many single star physics uncertainties, and while they
appear consistently in our accretor models, their physicality should
be tested further.

However, in the context of GW progenitors, blue loops are not crucial
since they correspond to a decrease in radius, which would not result
in binary interactions during the loop. They might change the
mass-loss history of the accretor, but since they occur in a short
evolutionary phase, their impact should be
limited. % Moreover, blue-loops are sensitive to
% the He profile above the H-burning shell, and specifically the mean
% molecular weight profile \citep{walmswell:2015, farrell:22}, thus it
% is not surprising that RLOF-accretion which modifies the stellar
% profile above the H-shell can lead to blue loops.

\begin{figure*}[htbp]
  \centering
  \includegraphics[width=\textwidth]{TAMS_profiles.pdf}
  \caption{Specific entropy (s, top row), H (bottom row, solid lines),
    and He (bottom row, dashed lines) TAMS profiles for non-rotating
    single stars (red), accretors (orange), and ``engineered'' models
    of the same total mass as the post-RLOF mass of the accretors. The
    overlapping gray bands emphasize the CEB region, which is well
    defined at TAMS.}
  \label{fig:TAMS_profiles}
  \script{TAMS_profiles.py}
\end{figure*}

\Figref{fig:TAMS_profiles} shows the entropy, H and He mass fraction
profiles at TAMS for our accretor models (orange), single non-rotating
stars (red) and ``engineered'' models of the same mass as the accretor
post-RLOF. We chose to compare our accretors to models of the same
total mass, since this quantity appears in the envelope binding energy
\citep[see \Eqref{eq:BE} and e.g.,][]{dekool:1990, dewi:2000}, and
rapid population synthesis codes typically use the total mass as the
main parameter to construct accretor models.
Nevertheless, it is not obvious that this is the most relevant
comparison: for instance, the (helium or carbon-oxygen) core mass is often
used to determine the final compact object \citep[e.g.,][]{fryer:2012,
  farmer:2019, patton:2021, renzo:2022, fryer:2022}, and comparing models of roughly the same
core-mass might be more appropriate (but is sensitive to the condition
defining the core edge). We present in
\Figref{fig:TAMS_profiles} a comparison between TAMS
profiles of accretors, single stars, and engineered models with the
same \emph{initial} mass as an alternative comparison that should bracket the
range of sensible comparison models.

Accretion through RLOF affects the CEB layer in more subtle ways than
we impose in our ``engineered'' models. One expects the CEB in
accretors to be steeper than in a star evolving as single, resulting
in models qualitatively more similar to our engineered models with the
steeper CEB (darker lines in
\Figref{fig:toy_models_example}-\ref{fig:rotation_models_example},
\Figref{fig:TAMS_profiles}-\ref{fig:grid_ratios}, and
\Figref{fig:TAMS_profiles_same_initial_mass}-\ref{fig:lambda_grid}). In fact, rejuvenation of the accretor is caused by an
increase in mass coordinate of the convective core (and consequent
refueling of H in the core): this is a thermal process related to
convection and convective boundary mixing. Since the convective
turnover timescale in the core of a massive star is shorter than
(\emph{i}) the nuclear timescale -- over which the composition of the
core changes, (\emph{ii}) the donor thermal timescale -- governing
case B RLOF mass-transfer, and (\emph{iii}) the accretor thermal
timescale -- governing the thermal readjustment of the star, the
increase in mass coordinate of the core should create a steep
composition and entropy gradient at the new (post-accretion) outer
edge. In practice, binary interactions occur after a finite amount of
the accretor's main-sequence has elapsed. This pre-RLOF duration
depends on the period and mass ratio of the two stars. For our
binaries with initial $P=100$\,days and $q=M_2/M_1\simeq 0.8$, RLOF
starts after $\sim{}$10, 9, and 5\,Myrs from the least massive to the
most massive binary, which correspond to central H mass fractions
$X_\mathrm{c}(^1\mathrm{H}) = 0.27, 0.23$, and 0.21 for the accretors.
Thus, the accreting stars are not homogeneous in composition at the
onset of RLOF: the convective core is already bound by a composition
gradient that generally impedes the mixing \citep[e.g.,][]{yoon:05}.
Thus, the advance in mass coordinate of the convective region due to mass accretion (i.e., the ``rejuvenation process'') during RLOF occurs against a gradient (of entropy,
$\mu$, H, and He) set by the pre-RLOF evolution.

To quantify the impact of the first RLOF phase on the outcome of the
second mass transfer phase, we evolve in time all the TAMS profiles
shown in \Figref{fig:TAMS_profiles} and compare them at fixed outer
radii. %  We define the local binding
% energy as:
% \todo{repetition}
% \begin{equation}
%   \label{eq:BE}
% BE(m, \alpha_{\rm th}) = - \int_{m}^M\,dm'\left( -\frac{G m'}{r(m')}+\alpha_\mathrm{th} u(m')\right)
% \end{equation}
% where we integrate from Lagrangian mass coordinate $m$ (assumed as the ``core''
% surviving a CE) to the surface, $u(m')$ is the internal energy of the
% spherical shell at mass coordinate $m'$ and thickness $dm'$, and $\alpha_{\mathrm{th}}=0,1$ is
% the fraction of the internal energy included
% \citep[e.g.,][]{dewi:2000}. For $\alpha_{\rm th}=0$, \Eqref{eq:BE}
% reduces to the gravitational potential energy except for the sign
% $BE(m, \alpha_{\rm th}=0)\equiv -E_\mathrm{grav}(m)$, plotted in
% \Figref{fig:toy_models_example} and \ref{fig:rotation_models_example}.
In \Figref{fig:BE_profiles}, we show the binding energy (solid lines,
including the internal energy, i.e. $\alpha_\mathrm{th}=1$) % and gravitational binding energy (dashed
% lines, $\alpha_\mathrm{th}=0$)
of our accretor models, single stars with initial mass roughly equal
to the accretors post-RLOF mass, and our engineered models (see also
\Figref{fig:lambda_grid} for the $\lambda_\mathrm{CE}$ profile defined in Appendix~\ref{sec:pop_synth_app}) The two
lowest mass accretors (left and central column) do not expand to
$R=1000\, R_\odot$ before carbon depletion. Generally speaking, the
accretors (orange) have lower binding energies than corresponding
single stars (red), and their profiles are qualitatively closer to the
engineered models with the steepest core (dark green curves), although
local deviations from this trend can occur for some $r$.

\begin{figure*}[hbtp]
  \includegraphics[width=\textwidth]{BE_profiles.pdf}
  \caption{Binding energy profile at fixed radii (right y-axis) as a
    function of radial mass coordinate. We only show profiles with
    $\alpha_\mathrm{th}=1$, that is accounting for the internal energy
    content of the star. Orange, red, and other colors show
    respectively the accretor models, single stars of same post-RLOF
    total mass, and engineered models with varying CEB steepness.
    Titles indicates the pre-RLOF and approximate post-RLOF accretor
    masses, and vertical gray dashed lines mark the total radius $R$
    of these models.}
  \label{fig:BE_profiles}
  \script{BE_profiles.py}
\end{figure*}

\subsection{Ratio of binding energies as a function of mass}

\begin{figure*}[htbp]
  \includegraphics[width=\textwidth]{grid_ratios.pdf}
  \caption{Ratios of the binding energy profiles (including internal
    energy) of the accretor stars divided stars of the same total mass
    post-RLOF. The red solid lines show the ratio to a non-rotating
    single star, while the other colors show the ratio to
    ``engineered'' star (see text). Each panel shows the ratios at the
    first time the models reach the radius indicated on the right and
    by the vertical dashed gray line.}
  \label{fig:grid_ratios}
  \script{grid_ratios.py}
\end{figure*}

Because of the large range of $BE$ across the stellar structures, it
is hard to appreciate the magnitude of the effect of RLOF-driven
rejuvenation on the structure of the stars in
\Figref{fig:BE_profiles}. Figure~\ref{fig:grid_ratios} presents the
ratio of the local value of the cumulative binding energy from the
surface of our accretor models divided the
comparison single stars, as a function of radius. To compute the
ratio, we interpolate linearly the single star models on the mesh of
our accretor, using the fractional Lagrangian mass coordinate $m/M$ as
independent coordinate. % , and we
% exclude the (small) mass regions where extrapolation would be needed
% due to the different total mass of the stars.
We calculate these ratio
when both the stars reach for the first time radii
$R=100, 200, 300, 500, 1000\,R_\odot$ (see vertical gray dashed
lines), corresponding to the assumed Roche lobe radius of the donor at
the onset of the CE.

In each panel, radial coordinates $r$ for which the lines in
\Figref{fig:grid_ratios} are below one correspond to radii at which
the accretor models are less bound than the comparison single
star or engineered model. In other words, it would take less energy to
eject the outer layers of the envelope of the accretors down to such
$r$. All of our accretor models, regardless of them being NS or BH
progenitor, and regardless of their evolutionary phase, produce
envelope structures ($r\gtrsim 10^{10}\,\mathrm{cm}$) that are less
bound than the corresponding non-rotating single star models (red),
and which are qualitatively more similar to the darker lines
representing engineered models with steeper CEB profiles.

The minimum ratio of binding energies occur roughly at the inner edge
of the CEB layer in \Figref{fig:grid_ratios}. Considering the
ratio to single stars (red lines), the minima range between 0.56-0.07;
0.58-0.08; and 0.51-0.04 from our least to most massive binary. In
other words, at the radius where the difference between accretors and
single stars models is largest, which is also the location where the
outcome of a common envelope is likely to be decided \citep[e.g.,][]{ivanova:2013}, the accretor's
binding energy is roughly between $\sim{}50-\mathrm{few}\%$ of the
binding energy of a single star. Regardless of the mass, the larger
the outer radius the smaller the minimum of the ratio of binding
energies: the differences caused by RLOF accretion and rejuvenation of
the core grows as the stars evolve and their core contract.

Defining the He core boundary as the outermost location where $X<0.01$
and $Y>0.1$, we can fix $m=M_\mathrm{He}$ in \Eqref{eq:BE} to obtain
an integrated binding energy for the envelope:
\begin{equation}
  \label{eq:BE_env}
  BE_\mathrm{env} \equiv BE(m=M_\mathrm{He}, \alpha_{th}=1) \ \ .
\end{equation}
\Figref{fig:BE_env_R} shows the evolution of this integrated envelope
binding energy as a function of the outer radius. Each panel shows one
of our binaries, from top to bottom: 36+30\,$M_\odot$,
20+17\,$M_\odot$, 18+15\,$M_\odot$. The dashed lines correspond to
single stars with the same total mass as the accretors post-RLOF,
while solid lines correspond to the accretors. For each binary, the
lower panel shows the ratios of the envelope binding energy of the
accretor divided the binding energy of the comparison single star
(i.e., the ratio of the solid lines to the dashed lines in the panel
above). To compute these ratios, we interpolate our accretor models
on the time-grid of the single stars using the central temperature
$\log_{10}(T_c/[\mathrm{K}])$ as independent coordinate. In each of
the lower panels the ratios are lower than one (marked by the gray
dashed lines) at almost all radii, again suggesting that post-RLOF,
accretor stars have envelopes that require less energy input to be
ejected in a CE event.

\begin{figure}[tp]
  \centering
  \includegraphics[width=0.5\textwidth]{BE_env_R.pdf}
  \caption{Evolution as a function of the photospheric radius $R$ of
    the binding energy (including the thermal
    energy, $\alpha_\mathrm{th}=1$) of the accretors and single stars
    of the same (post-RLOF) total mass. The bottom panels show the
    ratio, which is always smaller than 1 the first time a certain
    radius $R$ is reached, indicating that the
    accretors might have envelopes easier to unbind in a common
    envelope event.}
  \label{fig:BE_env_R}
  \script{BE_env_R.py}
\end{figure}

\section{Discussion \& Conclusions}
\label{sec:conclusions}

\todo{return to how conservative RLOF is}

We have modeled the impact of mass transfer on the envelope structure
of the accretor, focusing on thermal timescale, post-donor-main
sequence case B RLOF (see \Figref{fig:HRD}). The accretion of mass
drives the growth of the convective core, changing the core/envelope
boundary and ``rejuvenating'' the star. As the accretors evolve
beyond the main sequence, they experience large blue-loops which are
not expected in single stars of the same mass -- with potential
implications for asteroseismology \citep[e.g.,][]{dorn-wallenstein:20},
and the search for non-interacting companions to compact objects
\citep[e.g.,][]{breivik:17, andrews:19, chawla:21}.

% stellar hydro
The rejuvenation is driven by convective core boundary mixing
\citep[e.g.,][]{hellings:1983, hellings:1984, renzo:2021zoph}, and
does not occur in its absence \citep{braun:95}. The hydrodynamics of
convective boundaries in stellar regime is an active topic of research
\citep[e.g.,][]{anders:22a, anders:22b}, and observations of the width
of the main sequence \citep[e.g.,][]{brott:11} and asteroseismology
\citep[e.g.,][]{moravveji:16} suggest the presence of convective
boundary mixing in the core of massive main-sequence stars. In our
one-dimensional accretor models, the main mixing in the envelope is
Eddington-Sweet circulations and thermohaline mixing
\citep{renzo:2021zoph}, while at the core boundary is it overshooting,
with rotationally driven instabilities contributing to a lesser extent
during late RLOF. We adopt an exponentially decreasing overshooting
diffusion coefficient (\citealt{claret:17}) which may underestimate
the amount of mixing at the accretor core boundary. After RLOF, a
thick convective shell develops above the core see
\citep[see][]{renzo:2021zoph}, which also contributes to the different
binding energy profiles (see \Figref{fig:BE_profiles}).

% CE other uncertainties
We have focused on the consequences of the first mass transfer
accretion for next binary interaction on the path to a GW merger: the
CE event initiated by the RLOF-accretor. We find that accretors have
overall lower envelopes binding energy (both integrated from the
surface to the He core, see~\Figref{fig:BE_env_R} and as a function of
radius, see~\Figref{fig:BE_profiles}-\ref{fig:grid_ratios}). Since the
(second) common envelope in the evolution of isolated binary
progenitors to GW mergers occurs from the RLOF-accretor, the
systematically lower binding energy our accretor models compared to
single stars of the same outer radius may imply an easier to eject
envelope in the CE evolution of GW-progenitors.

Before the onset of the
dynamical instability in a CE event, a pre-CE thermal timescale phase
of mass transfer may occur \citep[e.g.,][]{hjellming:1987, pejcha:17,
  blagorodnova:2021}. This initial phase is usually neglected in
hydrodynamical simulations and may impact the envelope structure of
CE-donors whether they have previously been RLOF-accretors (as in our
models) or not.

% other applications
The structural change of the core/envelope boundary in the accretor
may also impact other aspects of its evolution. For example, even if
the following mass transfer is stable, the different binding energy
profile may affect the evolution during the second mass transfer. If
instead the binary is disrupted at the core-collapse of the RLOF-donor
\citep[e.g.,][]{blaauw:1961, renzo:2019walk}, the different chemical
composition at the surface \citep[e.g.,][]{blaauw:93, renzo:2021zoph}
may change the wind final velocity (but not the total mass loss rate),
with consequences for bow-shocks \citep[e.g.,][]{bodensteiner:18,
  moutzouri:22}. At core-collapse and explosion, the reverse shock
produced at the crossing of the He core boundary may be
different.\todo{check C shell in kippenhahns?}.

We have focused accretor models for progenitors of NS and BH. However,
the physical processes described should be similar in all accretor
stars with convective main sequence cores, down to initial mass
$\sim{}M_{\rm ZAMS}\gtrsim 1.2\,M_\odot$ \citep[see also][]{wang:20}.
Thus, also a fraction white dwarf progenitor binary systems, if
sufficiently massive and experiencing a (case B) RLOF phase of
evolution, may be influenced by the structural differences between
single stars and RLOF-accretors.

Our grid of models consists of only three systems, but could be
extended to inform a rapid-population synthesis semi-analytic
approximation for the binding energy of CE-donor that have accreted
mass in a previous stable mass transfer phase (see also \Figref{fig:lambda_grid}).


\software{MESA \citep{paxton:2011, paxton:2015, paxton:2018,
    paxton:2019}, pyMESA \citep{pymesa},
  compare\_workdir\_MESA\footnote{\url{https://github.com/mathren/compare_workdir_MESA/releases/tag/2.0}},
  Ipython \citep{ipython}, numpy
  \citep{numpy}, scipy \citep{scipy}, matplotlib \citep{matplotlib}, showyourwork
  \citep{showyourwork}.}

\appendix

\todo{smooth}
\section{Impact of core-envelope boundary and rotation on the binding  energy profile}
\label{sec:toy_models}

Before focusing on binary evolution models, we start by illustrating
with examples how the envelope binding energy depends on the CEB layer
(\Secref{sec:eng_examples}) and on the initial rotation rate of the
star (\Secref{sec:rot_examples}). Both can be significantly modified
by accretion during the first RLOF.


\begin{figure}[btp]
  \includegraphics[width=0.5\textwidth]{engineered_TAMS.pdf}
  \caption{TAMS entropy profile of a single 30\,$M_\odot$ star (red)
    and engineered models where we artificially modify the CEB region
    (gray shaded area partially overlapping for multiple models). The CEB for a
    single $30\,M_\odot$ star has a mass thickness of $5.81\,M_\odot$ at TAMS,
    while the engineered models span the range $\sim3-9.4\,M_\odot$.}
  \label{fig:engineered_TAMS}
  \script{engineered_TAMS.py}
\end{figure}

\Figref{fig:engineered_TAMS} shows an example grid of ``engineered
stars'' of $30M_\odot$. A gray shaded region highlights where each
model is modified (corresponding to H mass fraction
$X_\mathrm{c}(^1\mathrm{H})+0.01<X(^1\mathrm{H})<X_\mathrm{surf}(^1\mathrm{H})-0.01$,
with $X_c$ and $X_\mathrm{surf}$ the central and surface value of the
hydrogen mass fraction), and their overlap produces the shade in
\Figref{fig:engineered_TAMS}.

Starting from a non-rotating single star at TAMS (e.g., red model in
\Figref{fig:engineered_TAMS}), we modify the CEB specific entropy (s)
which controls the thermal properties of the gas, and its H, and He
profiles -- but do not change the mass fractions of other elements.
Specifically, we keep the same inner and outer profiles, but impose a
linear connection from the outer boundary of the H-depleted core to a
mass coordinate which we specify as a parameter (see
\Figref{fig:engineered_TAMS} and \Figref{fig:TAMS_profiles})
\todo{explain why accretors may look like this}. We let \mesa\ relax
the TAMS profiles to the desired entropy and composition profiles and
then recover thermal and hydrostatic equilibrium, and then evolve
until either carbon depletion or when these models first exceed
$1000\,R_\odot$.

\subsection{Steepness of the core-envelope boundary}
\label{sec:eng_examples}


\begin{figure}[bp]
  \centering
  \includegraphics[width=0.5\textwidth]{toy_models_example.pdf}
  \caption{The structure of the CEB at the end of the main sequence
    impacts the envelope binding energy profile throughout the
    remaining evolution. Dashed lines show the gravitational
    contribution only, while solid lines include the contribution of
    the internal energy. The red lines show a $30\,M_\odot$,
    non-rotating, $Z=0.0019$ model compared to ``engineered'' models
    of the same mass (see \Figref{fig:engineered_TAMS}), but
    artificially imposed profile at TAMS (other colors, see text). The
    top (bottom) axis indicates mass coordinate (radius). We compare
    the models when they first reach $500\,R_\odot$.}
  \label{fig:toy_models_example}
  \script{toy_models_example.py}
\end{figure}


\Figref{fig:toy_models_example} shows a comparison of the
gravitational and binding energy profiles of a $30\,M_\odot$ single
star (red solid line) to
``engineered'' models (see \Secref{sec:methods}), when stars first
reach radius $R=500\,R_\odot$. We
define the binding energy above mass coordinate $m$ as
\citep[e.g.,][]{dekool:1990, dewi:2000, lau:2022}:
\begin{equation}
  \label{eq:BE}
BE(m, \alpha_{\rm th}) = - \int_{m}^M\,dm'\left( -\frac{G m'}{r(m')}+\alpha_\mathrm{th} u(m')\right)
\end{equation}
% \begin{equation}
%   \label{eq:Egrav}
%   E_\mathrm{grav}(m) = \int_{m}^{M} - \frac{Gm'}{r(m')}\,dm' \ \ ,
% \end{equation}
with $M$ total mass of the star, $r(m')$ radius, $u(m')$ the internal
energy of a shell of mass thickness $dm'$ and outer Lagrangian mass
coordinate $m'$, and $G$ the gravitational constant. The integral
domain goes from mass coordinate $m$, which can be thought of the mass
of the ``core'' surviving a hypothetical CE, to the surface. The
parameter $0\leq \alpha_\mathrm{th}\leq 1$ is the fraction of internal
energy (including recombination energy) that can be used to lift the
shared CE. It is possible that $\alpha_\mathrm{th}$ may not be
constant during a CE (e.g., if recombination happens in already
unbound material it cannot contribute to the CE energetics) or across
binary systems entering a CE at different evolutionary stages. For
$\alpha_\mathrm{th}=0$, \Eqref{eq:BE} give the gravitational binding
energy (dashed lines in \Figref{fig:toy_models_example}), while
$\alpha_{\mathrm{th}}=1$ assumes perfectly fine tuned use of all the
internal energy (solid lines). These two cases bracket the range of
possible use of internal energy to eject the CE.

\Figref{fig:toy_models_example} shows that
binding energy depends on the structure of the CEB region. In single
stars, the CEB is determined by the extent of the convective boundary
mixing and the recession in mass coordinate of the
convective core. In \Figref{fig:toy_models_example}, lines of
different colors show a trend with shallower entropy and composition
profiles at TAMS (lighter curves in \Figref{fig:toy_models_example})
evolving into more bound inner envelopes (larger
binding energy inside $\log_{10}(r/\mathrm{cm})\lesssim 11.5$), and viceversa.

\subsection{Rotation}
\label{sec:rot_examples}

Mass transfer through RLOF from a binary companion also spins up the
accreting star, often to critical rotation\footnote{At critical
  rotation, the centrifugal force
  balances the gravitational pull at the equator, corresponding to
  angular frequency
  $\omega_\mathrm{crit}=\sqrt{(1-L/L_\mathrm{Edd})GM/R^3}$ is the
  critical spin frequency, with $L_\mathrm{Edd}$ the Eddington
  luminosity, and $L$ is the luminosity.} \citep[e.g.,][]{lubow:1975,
  packet:1981, cantiello:2007}. Although spinning up a star late during its
main-sequence evolution has a different structural consequences than
natal rotation (see \citealt{renzo:2021zoph}), our accretor models can
reach a rigid, close-to-critical rotation late during RLOF.
Thus, it is worth considering the impact of rotation on the envelope
structure and CEB by analyzing single star models rotating since
birth.

\begin{figure}[htbp]
  \centering
  \includegraphics[width=0.5\textwidth]{rotation_models_example.pdf}
  \caption{Same as \Figref{fig:toy_models_example} but comparing
    single stars differing by their initial rotation rate.}
  \label{fig:rotation_models_example}
  \script{rotation_models_example.py}
\end{figure}

Rotation has a two main effects: (\emph{i}) mixing can change the core
size directly (see \citealt{heger:2000, maeder:00}), (\emph{ii})
changes in the mass-loss rate \cite[e.g.,][]{langer:1998, muller:2014,
  gagnier:2019} affect the rate of recession of the convective core
\citep[e.g.,][]{renzo:2017, renzo:2020ppi_conv}. By inflating the
equatorial region, rotation changes the temperature and opacity
structure, and therefore the line-driving of the wind. Moreover,
rotation can have a dynamical effect, resulting in mass loss through
the combination of centrifugal forces and radiative pressure
($\Gamma-\Omega$ limit, \citealt{langer:1998}). One-dimensional stellar
evolution codes commonly assume that rotation increases the total mass
loss rate \citep[e.g.,][]{langer:1998, heger:2000} though this may not
always be true throughout the evolution \citep[e.g.,][]{gagnier:2019}.

\Figref{fig:rotation_models_example} shows the gravitational binding
energy profile of the single $30\,M_\odot$ star of
\Figref{fig:toy_models_example}, compared to single stars of the same
mass and varying initial $\omega/\omega_{\rm crit}$. % As
% \Figref{fig:rotation_models_example} illustrates, rotation also
% impacts the CEB layer: this is caused in part by
% rotational mixing affecting the CEB profile
% compared to non-rotating stars, and the impact on the wind mass loss
% rate and thus the mass and rate of recession of the convective core
% \citep[e.g.,][]{renzo:2017}.
For $\omega/\omega_\mathrm{crit}\lesssim 0.5$, corresponding to a
generous upper-bound for the typical birth rotation rate of single
massive stars \citep[e.g.,][]{ramirez-agudelo:2015}, the effect is
modest but non-negligible. For more extreme initial rotation rates
(achievable during RLOF, at least in the surface layers), the ratio of
the He core mass to total mass is significantly changed by rotational
mixing, which can result in larger binding energy differences than
changing the CEB layer at fixed core mass.


For stars accreting through RLOF in a binary both
effects illustrated in \Figref{fig:toy_models_example} and
\Figref{fig:rotation_models_example} act simultaneously, although the
timing and amplitude of the impact of mixing and rotation can be
different than for single stars \citep[e.g.,][]{renzo:2021zoph}.


\section{Comparison with same core mass}

\begin{figure*}[htbp]
  \centering
  \includegraphics[width=\textwidth]{TAMS_profiles_same_initial_mass.pdf}
  \caption{Specific entropy (top row), H (bottom row, solid lines),
    and He (bottom row, dashed lines) profiles for non-rotating single
    stars (red), accretors (orange), and ``engineered'' models of the
    same total mass as the ZAMS mass of the accretors. The overlapping
    gray bands emphasize the CEB region.}
  \label{fig:TAMS_profiles_same_initial_mass}
  \script{TAMS_profiles_same_initial_mass.py}
\end{figure*}


\section{Common envelope $\lambda_\mathrm{CE}$}
\label{sec:pop_synth_app}

\cite{dekool:1990} introduced a binding energy
parameter\footnote{While \cite{dekool:1990} used
  $\alpha_\mathrm{CE}=0$, we calculate $\lambda_\mathrm{CE}$ with
  $\alpha_\mathrm{CE}=1.0$, which provides a best case scenario for
  the ejection of the CE by harvesting the entire internal energy
  available in the gas.}
$\lambda_\mathrm{CE}$ to account for the internal structure of the
stars when calculating the post-CE orbit using energy conservation:
\begin{equation}
  \label{eq:lambda}
  \lambda_\mathrm{CE} \equiv \lambda_\mathrm{CE}(m) = (GM(M-m)/R)/BE(m, \alpha_\mathrm{th}=1.0) \ \ ,
\end{equation}
where again the Lagrangian mass coordinate $m$ can be interpreted as a
variable core mass \citep[see also][]{demarco:11, ivanova:2013}. We show
in \Figref{fig:lambda_grid} the $\lambda_\mathrm{CE}$ profiles for our models.

\begin{figure*}[htbp]
  \centering
  \includegraphics[width=\textwidth]{lambda_grid.pdf}
  \caption{Profile of the binding energy parameter
    $\lambda_\mathrm{CE}$ as a function of mass coordinate for
    accretors (orange), single stars (red), and our engineered stars
    (other colors) at selected total radii. The vertical dashed lines
    mark $M_\mathrm{He}$, that is the outermost location where
    $X<0.01$ and $Y>0.1$.}
  \label{fig:lambda_grid}
  \script{lambda_grid.py}
\end{figure*}

\bibliographystyle{aasjournal}
\bibliography{./CE_accretors.bib}
% % \bibliography{/home/math/Documents/Research/Biblio_papers/bibtex/zotero.bib}

\end{document}

%%% Local Variables:
%%% mode: latex
%%% TeX-master: t
%%% End:
